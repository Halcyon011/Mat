\documentclass[12pt,a4paper]{article}
\usepackage[utf8]{inputenc}
\usepackage[T2A]{fontenc}
\usepackage[russian]{babel}
\usepackage{amsmath,amssymb}
\usepackage{geometry}
\geometry{margin=2.5cm}
\title{Степенные ряды — задача 1}
\author{}
\date{}
\begin{document}
\maketitle
\textbf{Задача.} Найти область сходимости ряда
\[\sum_{n=0}^{\infty} \frac{2^n x^n}{(2n+1)^3}.\]
\bigskip
\textbf{Решение (базовыми методами).}

1) \textit{Общий член.}
\\[2mm]
(\(a_n(x)=\dfrac{2^n x^n}{(2n+1)^3}=\dfrac{(2x)^n}{(2n+1)^3}.\)\n

2) \textit{Радиус сходимости (корневой тест).}
\\[2mm]
\[\lim_{n\to\infty}\bigl|a_n(x)\bigr|^{1/n}=\lim_{n\to\infty}|2x|\cdot(2n+1)^{-3/n}=|2x|,\]
p\text{поскольку } (2n+1)^{3/n}\to1. \text{ Условие корневого теста: } |2x|<1, \text{ то есть } |x|<\tfrac{1}{2}. \text{ Следовательно, радиус сходимости } R=\tfrac{1}{2}.

3) \textit{Проверка концов интервала.}
\\[2mm]
\begin{itemize}
  \item При \(x=\tfrac{1}{2}\): \(a_n=\dfrac{1}{(2n+1)^3}\). Ряд сходится, так как это подпоследовательность p-ряда с \(p=3>1\).
  \item При \(x=-\tfrac{1}{2}\): \(a_n=\dfrac{(-1)^n}{(2n+1)^3}\). Ряд сходится абсолютно, так как \(\sum \dfrac{1}{(2n+1)^3}\) сходится.
\end{itemize}

\textbf{Итог.} Область сходимости: \([-\tfrac{1}{2},\, \tfrac{1}{2} ]\). Сходимость абсолютная на всём отрезке.

\end{document}